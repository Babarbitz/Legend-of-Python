\documentclass{article}
\usepackage{indentfirst}
\usepackage[dvipsnames]{xcolor}
\begin{document}

\title{Problem Statement}
\author{Group 1\\Giacomo Loparco | loparcog\\Bilal Jaffry | jaffryb\\Lucas Zacharewicz | zacharil}
\maketitle

\section{Problem}
Video games have become a large portion of the entertainment industry as of late, cementing themselves as a cornerstone
of many childhood memories. As time passes, technology advances, and a game which was once thought to be the peak of
technology becomes outdated, and the console it was produced for, obsolete. For many growing up, this game was The
Legend of Zelda, for the Nintendo Entertainment System, a video game console once considered a household item in the 
mid$-$80's, now a collector's item amongst fans alike. Due to it's outdated hardware being both hard and costly to
obtain, many will never be able to experience and relive this classic.

The goal of our group project is to develop and recreate the classic game The Legend of Zelda, using modern
techniques and iterating on past game design, to make this title available to the modern generation.

\section{Stakeholder Importance and Context}
The stakeholders to consider in this project range from those who have previously played and want to re-experience The Legend
of Zelda, to those who have never played it before, but are interested in the concept. This project would help stakeholders
to experience a classic game in a modern format, which in this sense would be Python, a very friendly language to both run
and modify. This allows many people interested in the game to be able to freely download, play, and customize
the game to their liking.

Our group also chose Python for this project as it is a very scalable language, allowing it to run on anything from
mobile phones to high-end desktop computers, allowing a variety of users to download and run the game on a variety of devices. 
\textcolor{blue}{The focus for this implementation will be with the desktop environment.}
The game will also be very lightweight, thanks to Python and the PyGame library,  allowing it to be run with minimal hardware limitations,
further enhancing the accessibility of the project.

\end{document}